% !TEX TS-program = pdflatex
% !TEX encoding = UTF-8 Unicode

% TeX-M (r1.0)
% For my math classes at UT Austin
% Notes template created by Abdon Morales for the College of Natural Science
% and for the Department of Mathematics and Computer Science
% (c) 2019 - 2024 Abdon Morales and the University of Texas at Austin
% This is a notes template for a LaTeX document using the "article" class for Mathematics (Calculus)
% at the University of Texas at Austin.

% Last change made: Jan 15, 2024 8:41 PM CST

% See "book", "report", "letter" for other types of document.

\documentclass[11pt]{article} % use larger type; default would be 10pt

% Start of Article customization options and addons (for more help and information reference to Overleaf's guides and docs on Latex.
\usepackage[utf8]{inputenc} % set input encoding (not needed with XeLaTeX)

%%% Examples of Article customizations
% These packages are optional, depending whether you want the features they provide.
% See the LaTeX Companion or other references for full information.

%%% PAGE DIMENSIONS
\usepackage{geometry} % to change the page dimensions
\geometry{letterpaper} % or letterpaper (US) or a5paper or....
% \geometry{margin=2in} % for example, change the margins to 2 inches all round
% \geometry{landscape} % set up the page for landscape
%   read geometry.pdf for detailed page layout information

\usepackage{graphicx} % support the \includegraphics command and options

% \usepackage[parfill]{parskip} % Activate to begin paragraphs with an empty line rather than an indent

%%% PACKAGES
\usepackage{booktabs} % for much better looking tables
\usepackage{array} % for better arrays (eg matrices) in maths
\usepackage{paralist} % very flexible & customisable lists (eg. enumerate/itemize, etc.)
\usepackage{verbatim} % adds environment for commenting out blocks of text & for better verbatim
\usepackage{subfig} % make it possible to include more than one captioned figure/table in a single float
\usepackage{exercise}
\usepackage{tcolorbox}
% Math tools
\usepackage{mathtools}
\usepackage{amsmath}
\usepackage{tikz} % For charts, mathematical graphs, and etc
%% Equal symbol for L'Hospital Rule
\newcommand\LR{\stackrel{\mathclap{\normalfont\mbox{L.R}}}{=}}
% These packages are all incorporated in the memoir class to one degree or another...

%%% HEADERS & FOOTERS
\usepackage{fancyhdr} % This should be set AFTER setting up the page geometry
\pagestyle{fancy} % options: empty , plain , fancy
\renewcommand{\headrulewidth}{0pt} % customise the layout...
\lhead{}\chead{}\rhead{}
\lfoot{}\cfoot{\thepage}\rfoot{}

%%% SECTION TITLE APPEARANCE
\usepackage{sectsty}
\allsectionsfont{\sffamily\mdseries\upshape} % (See the fntguide.pdf for font help)
% (This matches ConTeXt defaults)

%%% ToC (table of contents) APPEARANCE
\usepackage[nottoc,notlof,notlot]{tocbibind} % Put the bibliography in the ToC
\usepackage[titles,subfigure]{tocloft} % Alter the style of the Table of Contents
\renewcommand{\cftsecfont}{\rmfamily\mdseries\upshape}
\renewcommand{\cftsecpagefont}{\rmfamily\mdseries\upshape} % No bold!
%%% Question creation
\renewcommand\ExerciseName{Question~}
\renewcommand\AnswerName{Answer to question}
\renewcommand\ExerciseHeader{%
  \noindent\parbox[t]{.18\textwidth}{%
    \bfseries\large\ExerciseName\ExerciseHeaderNB\hfill}%
  \parbox[t]{.72\textwidth}{%
    \centering\bfseries\large%
    \ExerciseHeaderTitle\ExerciseHeaderOrigin}%
  \par\medskip
}
%%% END Article customizations

%%% The "real" document content comes below...

\title{Introduction to Linear Algebra and Syllabus Rundown}
\author{Abdon Morales \\ The University of Texas at Austin \\ M 340L \\ Kirk Blazek}
\date{\today \\ Lecture 1 \& 2}
%\date{} % Activate to display a given date or no date (if empty),
         % otherwise the current date is printed 

\begin{document}
\maketitle
\begin{tcolorbox}[width=\textwidth,colback={white},title={Important Dates for the course!!!},colbacktitle=yellow,coltitle=blue]
\begin{center}
\textbf{Midterm Exam - June 21st, 2024 (24-hour window, time limit: 1 hour, and 15 minutes)} \\
\textbf{Final Exam - July 12th, 2024 (3 hours and 40 minutes time limit)}
\end{center}
\end{tcolorbox}
\section*{Basic Terminology and notation}
$a_1 + a_2 x_2 + ... + a_n x_n = b$ (Linear equation) \\
$a_1, a_2, ..., a_n, b$ (known constants) \\
\begin{tcolorbox}[width=\textwidth,colback={white},title={System of linear equations},colbacktitle=yellow,coltitle=blue]
\begin{center}
$a_{1\;1} x_1 + a_{1\;2} x_2 + ... a_{1\;n} x_n = b_1$ \\
$a_{2\;1} x_1 + a_{2\;2} x_2 + ... a_{2\;n} x_n = b_2$ \\
\hrule
$a_{m\;1} x_1 + a_{m\;2} x_2 + ... a_{m\;n} x_n = b_m$
\end{center}
The solution to this system of equations is ($x_1, x_2, ..., x_n$) which satisfies every equation simultaneously.
\end{tcolorbox}

\subsection*{Example \#1}
\begin{align*}
2x + y &= -1 & 2(4 + 3y) + y &= -1 \\
x - 3y &= 4 & x &= 4 + 3y \uparrow
\end{align*}
Below we will solve for the first system in the system of equation; $2x + y = -1 \equiv 2(4 + 3y) + y = -1$
\begin{align*}
2(4 + 3y) + y &= -1 \\
8 + 6y + y &= -1 \\
7y &= -9 \\
y &= - \frac{9}{7}
\end{align*}
We then plug in $y$ into our $x=$ system to solve for $x$.
\begin{align*}
x &= 4 + 3y \\
x &= 4 + 3(-\frac{9}{7}) \\
&= 4 - \frac{27}{7} \\
&= \frac{1}{7} \\
\end{align*}
\textbf{Note!} A \textbf{\textit{solution set}} is the set of all solutions to a system

For this example, our solution set would be $\{ (x,y)\in \R^2 |\:x - y = 3 \}$

\section*{What is a matrix?}
A matrix is a rectangular grid of number; it can also mean "womb" in Latin.
\begin{equation}
A =
\begin{bmatrix}
a_{1\;1} & a_{1\;2} & ... & a_{1\;n} \\
a_{2\;1} & a_{2\;2} & ... & a_{1\;n} \\
... & ... & ... & ... \\
a_{n\;1} & ... & ... & a_{m\;n}
\end{bmatrix}
\end{equation}

$a_{\text{ij}}$ means the element in the ith row and the jth column.

\subsection*{Matrix anatomy}
\begin{equation}
A =
\begin{bmatrix}
2 & 0 \\
-3 & 8 \\
5 & 7 \\
\end{bmatrix}
\end{equation}
$a_{3\;2} = 7$, while $a_{2\;3} = \text{not defined}$. The size of the matrix is $\text{size}=3\times 2 $ matrix.

The size of a matrix $&= m \times x \\ &= \text{num rows} \times \text{num columns}$

\textbf{Proof of a matrix}
\begin{align*}
a_{1\;1} x_1 + a_{1\;2} x_2 + ... + a_{1\;n} x_n &= b_1 \\
a_{2\;1} x_1 + a_{2\;2} x_2 + ... + a_{2\;n} x_{2} &= b_2 \\
a_{m\;1} x_1 + ... + a_{m\;n} x_n &= b_m \\
A=
\begin{bmatrix}
a_{1\;1} & a_{1\;2} & ... & a_{1\;n} \\
a_{2\;1} & a_{2\;2} & ... & a_{2\;n} \\
... & ... & ... & ... \\
a_{m\;1} & a_{m\;2} & ... & a_{m\;n}
\end{bmatrix}
\leftarrow \text{coefficent matrix} \\
A=
\begin{bmatrix}
a_{1\;1} & ... & a_{1\;n} & b_1 \\
. & . & . & .\\
. & . & . & . \\
. & . & . & . \\
a_{m\;1} & ... & a_{m\;n} & b_m \\
\end{bmatrix}
\leftarrow \text{augmented matrix}
\end{align*}

\section*{Elementary Row Operations (June 7th)}
\subsection*{Example \#1}
\begin{align*}
-2(x + 3y &= 1) \rightarrow -2x - 6y &= -2 \\
2x + 5y &= 1 & 2x + 5y &=1 \\
x + 3 &= 1 & -y = -1 \\
x =& -2 & y =& 1
\end{align*}
\textbf{Solution:} \\
Solution coordinates: $(-2 , 1)$\\
The matrices below are augmented matrices!s
\begin{align*}
\begin{bmatrix}
1 & 0 & -2 \\
0 & 1 & 1
\end{bmatrix}
\begin{bmatrix}
1 & 3 & 1 \\
2 & 5 & 1
\end{bmatrix}
\end{align*}
The solution matrices are know as \textbf{\textit{equivalent systems}} which means \textit{they share the same solution set}

\subsection*{Reduced Row Reduction}
\begin{enumerate}
\item Swap the two rows
\item Multiply a row by a non-zero number
\item Replace a row by the sum of that row and a multiple of another row
\end{enumerate}

\begin{align*}
\text{This linear system is not allowed to be multiplied simultaneously} \\
-3(2x + 7y = 15) \\
2(3x - 9y = 84) \\
\end{align*}
\begin{gather*}
\textbf{The goal} \\
\begin{bmatrix}
1 & 3 & 1 \\
2 & 5 & 1
\end{bmatrix} \rightarrow
\begin{bmatrix}
1 & 0 & -2 \\
0 & 1 & 1
\end{bmatrix}
\end{gather*}

\subsubsection*{Example \#1}
\begin{gather*}
\text{This is an augmented matrix (matrices)}\\
\begin{bmatrix}
1 & 3 & 1\\
2 & 5 & 1
\end{bmatrix} \rightarrow R_2 - 2R_1
\begin{bmatrix}
1 & 3 & 1\\
0 & -1 & -1
\end{bmatrix} \rightarrow -R_2
\begin{bmatrix}
1 & 3 & 1\\
0 & 1 & 1
\end{bmatrix} R_1 - 3R_2
\end{gather*}
\textbf{Multiply the row that is changing}\\
\begin{gather*}
\begin{bmatrix}
1 & 0 & -2\\
0 & 1 & 1
\end{bmatrix} 
\begin{align*}
x = -2 \\ y = 1
\end{align*}
\end{gather*}


\subsubsection*{Example \#2 (Complex System)}
% Fix this dog crap
\begin{gather*}
\begin{align*}
-4x -5y - 6z &= -3\\
-2x -3y -3z &= -2\\
5x + 7y + 8z &= 5\\
\end{align*} \rightarrow 
\begin{bmatrix}
-4 & -5 & -6 & -3\\
-2 & -3 & -3 & -2\\
5 & 7 & 8 & 5
\end{bmatrix} \rightarrow
\begin{bmatrix}
1 & 0 & 0 & ?\\
0 & 1 & 0 & ?\\
0 & 0 & 1 & ?
\end{bmatrix}
\end{gather*}\\
\begin{gather*}
\begin{bmatrix}
-4 & -5 & -6 & -3\\
-2 & -3 & -3 & -2\\
5 & 7 & 8 & 5
\end{bmatrix} R_1 + R_3 \rightarrow
\begin{bmatrix}
1 & 2 & 2 & 2\\
-2 & -3 & -3 & -2\\
5 & 7 & 8 & 5
\end{bmatrix} 
\begin{gather*}
R_2 + 2R_1\\R_3-5R_1
\end{gather*}
\begin{bmatrix}
1 & 2 & 2 & 2\\
0 & 1 & 1 & 2\\
0 & -3 & -2 & -5
\end{bmatrix} R_3 + 3R_2
\begin{bmatrix}
1 & 2 & 2 & 2\\
0 & 1 & 1 & 2\\
0 & 0 & 1 & 1
\end{bmatrix} \leftarrow \text{This is trangular form}\\
\end{gather*}

\begin{align*}
x + 2y + 2z &= 2\\
y + z &= 2\\
z = 1, x = -2, y = 1
\end{align*}
\subsection*{(Row) Echelon Form of a Matrix}
\begin{enumerate}
\item The leading number (1st non-zero number going left to right) of any row is to the right of the leading entries above it.
\item Everything in a column below a leading entry is zero.
\item Any row of zeros is below all non-zeros rows.
\end{enumerate}

\subsubsection*{Example \#1}
\begin{bmatrix}
2 & 3 & 8 & -7 & 5\\
0 & 0 & 3 & 6 & 4\\
0 & 0 & 0 & 2 & 1
\end{bmatrix} \leftarrow \text{IN \textbf{echelon form!}}\\
\begin{bmatrix}
5 & 7 & 3 & 1 & 2\\
0 & 2 & 4 & 8 & -1\\
0 & 3 & 6 & 5 & 9
\end{bmatrix} \leftarrow \text{\textbf{NOT} in echelon form!}\\

\subsection*{Reduced (Row) Echelon Form [Optimal]}
\textit{Steps 1-3 from REF}
\begin{enumerate}
\item Every leading entry is 1
\item Everything in a column above a leading entry is zero
\end{enumerate}
\end{document}
