% !TEX TS-program = pdflatex
% !TEX encoding = UTF-8 Unicode

% TeX-M (r1.0)
% For my math classes at UT Austin
% Notes template created by Abdon Morales for the College of Natural Science
% and for the Department of Mathematics and Computer Science
% (c) 2019 - 2024 Abdon Morales and the University of Texas at Austin
% This is a notes template for a LaTeX document using the "article" class for Mathematics (Calculus)
% at the University of Texas at Austin.

% Last change made: Jan 15, 2024 8:41 PM CST

% See "book", "report", "letter" for other types of document.

\documentclass[11pt]{article} % use larger type; default would be 10pt

% Start of Article customization options and addons (for more help and information reference to Overleaf's guides and docs on Latex.
\usepackage[utf8]{inputenc} % set input encoding (not needed with XeLaTeX)

%%% Examples of Article customizations
% These packages are optional, depending whether you want the features they provide.
% See the LaTeX Companion or other references for full information.

%%% PAGE DIMENSIONS
\usepackage{geometry} % to change the page dimensions
\geometry{letterpaper} % or letterpaper (US) or a5paper or....
% \geometry{margin=2in} % for example, change the margins to 2 inches all round
% \geometry{landscape} % set up the page for landscape
%   read geometry.pdf for detailed page layout information

\usepackage{graphicx} % support the \includegraphics command and options

% \usepackage[parfill]{parskip} % Activate to begin paragraphs with an empty line rather than an indent

%%% PACKAGES
\usepackage{booktabs} % for much better looking tables
\usepackage{array} % for better arrays (eg matrices) in maths
\usepackage{paralist} % very flexible & customisable lists (eg. enumerate/itemize, etc.)
\usepackage{verbatim} % adds environment for commenting out blocks of text & for better verbatim
\usepackage{subfig} % make it possible to include more than one captioned figure/table in a single float
\usepackage{exercise}
% Math tools
\usepackage{mathtools}
\usepackage{amsmath}
\usepackage{amssymb}
\usepackage{tikz} % For charts, mathematical graphs, and etc
%% Equal symbol for L'Hospital Rule
\newcommand\LR{\stackrel{\mathclap{\normalfont\mbox{L.R}}}{=}}
% These packages are all incorporated in the memoir class to one degree or another...

%%% HEADERS & FOOTERS
\usepackage{fancyhdr} % This should be set AFTER setting up the page geometry
\pagestyle{fancy} % options: empty , plain , fancy
\renewcommand{\headrulewidth}{0pt} % customise the layout...
\lhead{}\chead{}\rhead{}
\lfoot{}\cfoot{\thepage}\rfoot{}

%%% SECTION TITLE APPEARANCE
\usepackage{sectsty}
\allsectionsfont{\sffamily\mdseries\upshape} % (See the fntguide.pdf for font help)
% (This matches ConTeXt defaults)

%%% ToC (table of contents) APPEARANCE
\usepackage[nottoc,notlof,notlot]{tocbibind} % Put the bibliography in the ToC
\usepackage[titles,subfigure]{tocloft} % Alter the style of the Table of Contents
\renewcommand{\cftsecfont}{\rmfamily\mdseries\upshape}
\renewcommand{\cftsecpagefont}{\rmfamily\mdseries\upshape} % No bold!
%%% Question creation
\renewcommand\ExerciseName{Question~}
\renewcommand\AnswerName{Answer to question}
\renewcommand\ExerciseHeader{%
  \noindent\parbox[t]{.18\textwidth}{%
    \bfseries\large\ExerciseName\ExerciseHeaderNB\hfill}%
  \parbox[t]{.72\textwidth}{%
    \centering\bfseries\large%
    \ExerciseHeaderTitle\ExerciseHeaderOrigin}%
  \par\medskip
}
%%% END Article customizations

%%% The "real" document content comes below...

\title{Vectors in $\mathbb{R}^n$}
\author{Abdon Morales \\ The University of Texas at Austin \\ M 340L \\ Kirk Blazek}
\date{June 10, 2024 \\ Lecture 3}
%\date{} % Activate to display a given date or no date (if empty),
         % otherwise the current date is printed 

\begin{document}
\maketitle

A column vector is an $n \times 1$ matrix
\begin{gather*}
\text{Linear Algebra} & \text{Physics}\\
\vec{x}=
\begin{bmatrix}
x_1\\
x_2\\
...\\
x_n
\end{bmatrix} & \vec{v} = \{ v_1, v_2, v_3 \}
\end{gather*}
\section*{Vector Addition}
If $\vec{u}, \vec{v} \in \mathbb{R}^n$, then $\vec{u} + \vec{v}$ is the vector in $\mathbb{R}^n$ whose ith element is $u_i + v_i$.
\begin{equation}
\vec{u} + \vec{v} =
\begin{bmatrix}
u_1\\
u_2\\
...\\
u_n
\end{bmatrix} +
\begin{bmatrix}
v_1\\
v_2\\
...\\
v_n
\end{bmatrix} =
\begin{bmatrix}
u_1 + v_1\\
u_2 + v_2\\
...\\
u_n + v_n
\end{bmatrix}
\end{equation}
\subsection*{Example \#1}
\begin{gather*}
\vec{u} + \vec{v} =
\begin{bmatrix}
2\\
1
\end{bmatrix} +
\begin{bmatrix}
-3\\
3
\end{bmatrix} =
\begin{bmatrix}
-1\\
4
\end{bmatrix}
\end{gather*}
The vector addition below cannot be allowed and will be seen as not defined or in other words undefined,
\begin{align*}
\begin{bmatrix}
2\\
-3\\
7
\end{bmatrix} +
\begin{bmatrix}
5\\
1
\end{bmatrix} &= \text{\textbf{not defined}}\\
+ 8 &= \text{\textbf{not defined}}\\
\end{align*}

\section*{Scalar Multiplication}
If $\vec{u} \in \mathbb{R}^n$ and $c \in \mathbb{R}$, then the scalar multiple of $c$ and $\vec{u}$, written as $c\vec{u}$ is the vector in $\mathbb{R}^n$ whose ith component is $cu_i$.

\begin{equation}
c\vec{u} = c
\begin{bmatrix}
u_1\\
u_2\\
...\\
u_n
\end{bmatrix} =
\begin{bmatrix}
cu_1\\
cu_2\\
...\\
cu_n
\end{bmatrix}
\end{equation}
If $\{ \vec{v}_1, \vec{v}_2, ..., \vec{v}_k \{ $ is a set of vectors in $\mathbb{R}^n$ and $\{ c_1, c_2, ..., c_k \}$ is a set of scalars in $\mathbb{R}$, then
\begin{equation}
c_1\vec{v}_1 + c_2\vec{v}_2 + c_3\vec{v}_3 + ... + c_k\vec{v}_k
\end{equation}
is the linear combination of the $\vec{v}$'s using the $c$'s as weights.\\

If $\{ \vec{v}_1, \vec{v}_2, ..., \vec{v}_k \} $ is a set of vectors in $\mathbb{R}^n$, then the span $s = \text{span}\{ \vec{v}_1, ..., \vec{v}_k\} $ is the set of \textbf{\underline{ALL}} linear combination of these vectors.

\subsection*{Example #2}
\begin{equation}
\text{span}\{
\begin{bmatrix}
1\\
1\\
1
\end{bmatrix}, 
\begin{bmatrix}
1\\3\\0
\end{bmatrix}\} =
\{ c_1
\begin{bmatrix}
1\\1\\1
\end{bmatrix} + c_2
\begin{bmatrix}
1\\3\\0
\end{bmatrix} | c_1, c_2 \in \mathbb{R}\}
\end{equation}

\subsection*{Problem Set 1}
\begin{align*}
a_{1\;1} x_1 + a_{1\;2} x_2 + ... + a_{1\;n} x_n &= b_1 \\
a_{2\;1} x_1 + a_{2\;2} x_2 + ... + a_{2\;n} x_{2} &= b_2 \\
a_{m\;1} x_1 + ... + a_{m\;n} x_n &= b_m
\end{align*}
If $A$ is an $m \times x$ matrix and $\vec{x}$ is a vector in $\mathbb{R}^n$ then $A\vec{x}$ is the vector in $\mathbb{R}^n$ which is the linear combination of the columns of $A$ using the components of \vec{x} as weights.
\begin{equation}
A\vec{x}=
\begin{bmatrix}
\vec{a}_1 & \vec{a}_2 & ... & \vec{a}_n
\end{bmatrix}
\begin{bmatrix}
x_1\\x_2\\...\\x_n
\end{bmatrix} = x_1\vec{a}_1 + x_2\vec{a}_2 + ... + x_na_n
\end{equation}
\subsubsection*{Problem 1}
\begin{align*}
\begin{bmatrix}
-1 & 3 & 2 & 0\\4 & 1 & 0 & -2\\
0 & 1 & 3 & 1
\end{bmatrix}
\begin{bmatrix}
2\\1\\-1\\1
\end{bmatrix} &= 2
\begin{bmatrix}
-1\\4\\0
\end{bmatrix} + 1
\begin{bmatrix}
3\\1\\1
\end{bmatrix} - 1
\begin{bmatrix}
2\\0\\3
\end{bmatrix} + 1
\begin{bmatrix}
0\\-2\\1
\end{bmatrix}\\
&=
\begin{bmatrix}
-2\\8\\0
\end{bmatrix} +
\begin{bmatrix}
3\\1\\1
\end{bmatrix} +
\begin{bmatrix}
-2\\0\\-3
\end{bmatrix} +
\begin{bmatrix}
0\\-2\\1
\end{bmatrix}\\
&=
\begin{bmatrix}
-1\\7\\-1
\end{bmatrix}
\end{align*}

* can be written as $A\vec{x} = b$, can be written as a linear combination of the columns of A?

If $A=
\begin{bmatrix}
\vec{a}_1 & \vec{a}_2 & ... & \vec{a}_2
\end{bmatrix}
$, is $\vec{b}$ in $\text{span}\{ \vec{a}_1, \vec{a}_2, ..., \vec{a}_n\}$?

% Add the example here, for this definition

Pivot in every row A\vec{x}=\vec{b} will always have a solution.

If $A$ is an $m \times n$ matrix, then the following are equivalent.
\begin{itemize}
	\item $\forall \vec{b} \in \mathbb{R}^m$, $A\vec{x}=b$ has a solution
	\item Each $\vec{b} \in \mathbb{R}^m$ is a linear combination of the columns of A
	\item The columns of $A$ span $\mathbb{R}^m
	\item $A$ has a pivot in every row
\end{itemize}
\end{document}
