% !TEX TS-program = pdflatex
% !TEX encoding = UTF-8 Unicode

% TeX-M (r1.0)
% For my math classes at UT Austin
% Notes template created by Abdon Morales for the College of Natural Science
% and for the Department of Mathematics and Computer Science
% (c) 2019 - 2024 Abdon Morales and the University of Texas at Austin
% This is a notes template for a LaTeX document using the "article" class for Mathematics (Calculus)
% at the University of Texas at Austin.

% Last change made: Jan 15, 2024 8:41 PM CST

% See "book", "report", "letter" for other types of document.

\documentclass[11pt]{article} % use larger type; default would be 10pt

% Start of Article customization options and addons (for more help and information reference to Overleaf's guides and docs on Latex.
\usepackage[utf8]{inputenc} % set input encoding (not needed with XeLaTeX)

%%% Examples of Article customizations
% These packages are optional, depending whether you want the features they provide.
% See the LaTeX Companion or other references for full information.

%%% PAGE DIMENSIONS
\usepackage{geometry} % to change the page dimensions
\geometry{letterpaper} % or letterpaper (US) or a5paper or....
% \geometry{margin=2in} % for example, change the margins to 2 inches all round
% \geometry{landscape} % set up the page for landscape
%   read geometry.pdf for detailed page layout information

\usepackage{graphicx} % support the \includegraphics command and options

% \usepackage[parfill]{parskip} % Activate to begin paragraphs with an empty line rather than an indent

%%% PACKAGES
\usepackage{booktabs} % for much better looking tables
\usepackage{array} % for better arrays (eg matrices) in maths
\usepackage{paralist} % very flexible & customisable lists (eg. enumerate/itemize, etc.)
\usepackage{verbatim} % adds environment for commenting out blocks of text & for better verbatim
\usepackage{subfig} % make it possible to include more than one captioned figure/table in a single float
\usepackage{exercise}
% Math tools
\usepackage{mathtools}
\usepackage{amsmath}
\usepackage{tikz} % For charts, mathematical graphs, and etc
%% Equal symbol for L'Hospital Rule
\newcommand\LR{\stackrel{\mathclap{\normalfont\mbox{L.R}}}{=}}
% These packages are all incorporated in the memoir class to one degree or another...

%%% HEADERS & FOOTERS
\usepackage{fancyhdr} % This should be set AFTER setting up the page geometry
\pagestyle{fancy} % options: empty , plain , fancy
\renewcommand{\headrulewidth}{0pt} % customise the layout...
\lhead{}\chead{}\rhead{}
\lfoot{}\cfoot{\thepage}\rfoot{}

%%% SECTION TITLE APPEARANCE
\usepackage{sectsty}
\allsectionsfont{\sffamily\mdseries\upshape} % (See the fntguide.pdf for font help)
% (This matches ConTeXt defaults)

%%% ToC (table of contents) APPEARANCE
\usepackage[nottoc,notlof,notlot]{tocbibind} % Put the bibliography in the ToC
\usepackage[titles,subfigure]{tocloft} % Alter the style of the Table of Contents
\renewcommand{\cftsecfont}{\rmfamily\mdseries\upshape}
\renewcommand{\cftsecpagefont}{\rmfamily\mdseries\upshape} % No bold!
%%% Question creation
\renewcommand\ExerciseName{Question~}
\renewcommand\AnswerName{Answer to question}
\renewcommand\ExerciseHeader{%
  \noindent\parbox[t]{.18\textwidth}{%
    \bfseries\large\ExerciseName\ExerciseHeaderNB\hfill}%
  \parbox[t]{.72\textwidth}{%
    \centering\bfseries\large%
    \ExerciseHeaderTitle\ExerciseHeaderOrigin}%
  \par\medskip
}
%%% END Article customizations

%%% The "real" document content comes below...

\title{Linear Independence and Dependence}
\author{Abdon Morales \\ The University of Texas at Austin \\ M 340L \\ Dr. Kirk Blazek}
\date{June 11, 2024 \\ Lecture 5}
%\date{} % Activate to display a given date or no date (if empty),
         % otherwise the current date is printed 

\begin{document}
\maketitle

If $s=\{ \vec{v})_1, \vec{v}_2, ..., \vec{v}_k\}$ is a set of vectors in $\mathbb{R}^n$, then $s$ is linearly independent if
\begin{equation}
c_1\vec{v}_1 + c_2\vec{v}_2 + ... + c_k\vec{v}_k = \vec{0}
\end{equation}
can only be solved when $c_1 = c_2 = ... = c_k = 0$

The set $s$ is linearly dependent if the equation $c_1\vec{v}_1 + c_2\vec{v}_2 + ... + c_k\vec{v}_k = \vec{0}$ can be solved where at least one of the $c$'s is not zero.

\subsection*{Example \#1}
Let's pretend $\{ \vec{v}_1, ..., \vec{v}_k \}$ is linearly dependent and $c_k \neq 0$
\begin{gather*}
c_k\vec{v}_k = -c_1\vec{v}_1-c_2\vec{v}_2-...-c_{k-1}\vec{v}_{k-1} \\
\vec{v}_k = -\frac{c_1}{c_k}\vec{v}_1 - \frac{c_2}{c_k}\vec{v}_2 - ... - \frac{c_{k-1}}{c_k}\vec{v}_{k-1}\\
\uparrow \text{Linear dependence means (at least) one vector can be written as a linear} \\
\text{combination of the other vectors.}

\{ \vec{v}_1, \vec{v}_2, ..., \vec{v}_k\} \rightarrow \text{linearly independent} \lor \text{linearly dependent}

c_1\vec{v}_1 + c_2\vec{v}_2 +...+c_k\vec{v}_k = \vec{0}\\

\begin{bmatrix}
\vec{v}_1 & \vec{v}_2 & ... & \vec{v}_k
\end{bmatrix}
\begin{bmatrix}
c_1 \\
c_2 \\
... \\
c_3
\end{bmatrix} = \vec{0} \rightarrow A\vec{x} = \vec{0}
\end{gather*}

$\{ \vec{v}_1, \vec{v}_2, ..., \vec{v}_k\}$ is a linear independence set of vectors 
$\Longleftrightarrow A\vec{x} = \vec{0}$ where $A = 
\begin{bmatrix}
\vec{v}_1 & \vec{v}_2 & ... & \vec{v}_k
\end{bmatrix}
$ only has the trivial solution
$\Longleftrightarrow A$ has a pivot in every column

$\{ \vec{v}_1, \vec{v}_2, ..., \vec{v}_k\}$ is a linear dependent set of vectors $\longleftrightarrow A\vec{x} = \vec{0}$ has a non-trivial solutions $\longleftrightarrow$ there is at least one free variable in $A\vec{x} = \vec{0} \longleftrightarrow A$ does not have pivot in every column.

\begin{gather*}
\{\begin{bmatrix}
1\\-3\\2
\end{bmatrix}, \begin{bmatrix}
5\\1\\7
\end{bmatrix}, \begin{bmatrix}
0\\0\\0
\end{bmatrix}
\} \; c_1\vec{v}_1 + c_2\vec{v}_2 + c_3\vec{v}_3 = \vec{0}
\end{gather*}
If 0 is in your set $\rightarrow$ is linearly dependent

% insert example here

\begin{gather*}
\{ \vec{v}_1, \vec{v}_2\} \\
c_1\vec{v}_1 + c_2\vec{2} = \vec{0}
\text{if} c_2 \neq 0, \vec{v}_2 = \frac{-c_1}{c_2}\vec{v}_1
\end{gather*}
two vectors are linearly dependent $\longleftrightarrow$ they are parallel i.e one is a scalar multiple of the other.
\begin{equation}
\{\begin{bmatrix}
2\\7\\-5
\end{bmatrix}, \begin{bmatrix}
6\\21\\-15
\end{bmatrix}\} \rightarrow \text{linear dependence}
\end{equation}
\begin{equation}
\{\begin{bmatrix}
2\\7\\-5
\end{bmatrix}, \begin{bmatrix}
6\\21\\4
\end{bmatrix}\} \rightarrow \text{linear independence}
\end{equation}

% Add example here

% Add example here

\end{document}
